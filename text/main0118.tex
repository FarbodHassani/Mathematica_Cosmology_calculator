\documentclass[12pt,a4paper]{article}
%\usepackage[allfiguresdraft]{draftfigure}
%\newcommand{\pdfextension}{pdf}
%\newcommand{\pngextension}{png}
\usepackage{cite}
%\usepackage{natbib}
\usepackage{breakcites}
\usepackage[normalem]{ulem}
\usepackage{comment}
\usepackage[font=footnotesize,labelfont=bf]{caption}
\let\rho=\varrho
\def\fref#1{Fig.~\ref{#1}}
\def\tabref#1{Table~\ref{#1}}
\def\cref#1{Condition~\ref{#1}}
\def\Cref#1{Corollary~\ref{#1}}
\def\eref#1{(\ref{#1})}
\def\sref#1{Section \ref{#1}}
\def\aref#1{Appendix \ref{#1}}
\def\lref#1{Lemma~\ref{#1}}
\def\rref#1{Remark~\ref{#1}}
\def\tref#1{Theorem~\ref{#1}}
\def\dref#1{Definition~\ref{#1}}
\def\pref#1{Proposition~\ref{#1}}
\def\qref#1{Subsection~\ref{#1}}

\def\dref#1{Definition~\ref{#1}}
\def\myundefined#1{\special{ps: 1 0 0 setrgbcolor}{what is #1?}%
	\special{ps: 0 0 0 setrgbcolor}}
\newenvironment{myitem}
{\begin{itemize}
  \setlength{\itemsep}{1pt}
  \setlength{\parskip}{0pt}
  \setlength{\parsep}{0pt}}
{\end{itemize}}

\newenvironment{myenum}
{\begin{enumerate}
  \setlength{\itemsep}{1pt}
  \setlength{\parskip}{0pt}
  \setlength{\parsep}{0pt}}
{\end{enumerate}}

%\usepackage{sub_JP}
%\usepackage{fancyhdr}
%\pagestyle{fancy}
\usepackage{amsmath}
\usepackage{amsfonts}
\usepackage{graphicx}
%\usepackage{makeidx}
\usepackage{times}
\usepackage{amsthm}
\usepackage{ amssymb }
\usepackage{color}
\usepackage{mhequ}
\usepackage{dsfont}
%\usepackage[scanall]{psfrag}
\usepackage[margin=2.5cm]{geometry}
\usepackage{color}
\usepackage{url}
\usepackage{lipsum}
%\usepackage{authblk}
\usepackage{subfig}
\usepackage{mhequ}

\usepackage{tikz}
\usepackage[graphics, active, tightpage]{}



%\usepackage[expansion=true]{microtype}

%\captionsetup[figure]{margin=2cm,font=footnotesize,labelfont=bf,labelsep=endash,textfont=rm}\captionsetup[subfigure]{margin=0pt}


%\begin{equa}
% \dot\phi(t) &\sim -2\gamma \epsilon ^{2n-1}~,\\
% t\dot\phi(t)+\dot \theta(t) &\sim 0~,\\
%   \norm{z(t)}&\text{ remains bounded by }\OO(\epsilon ^n)~.
%\end{equa}
%
%
%
%this will make a label , vertically centered
%you can also say \begin{equa}\\label{eq:main_real}
%
%\begin{equa}[eq:main_real]
% \dot q_j &= -\epsilon (\Delta p)_j
% -p_j+(q_j^2+p_j^2)p_j-\delta_ {j,n}\gamma\epsilon q_n~,\\
% \dot p_j &= \hphantom{-}\epsilon (\Delta q)_j +q_j-(q_j^2+p_j^2)q_j-\delta_
%{j,n}\gamma\epsilon p_n~~.
%\end{equa}


\def\thecomma{\ifx,\thenewxt \else\ifx;\thenext \else\ifx.\thenext
	\else\ifx!\thenext \else\ifx:\thenext\else\ifx)\thenext \else \
	\fi\fi\fi\fi\fi\fi}
\def\condblank{\futurelet\thenext\thecomma}
\def\ie{{\it i.e.,}\condblank}
\def\eg{{\it e.g.,}\condblank}

\numberwithin{equation}{section}



\newtheorem*{oneshot}{Hypothesis~\ref{h:assumption}}
\newtheorem{theorem}{Theorem}[section]
\newtheorem{lemma}[theorem]{Lemma}
\newtheorem{proposition}[theorem]{Proposition}
\newtheorem{definition}[theorem]{Definition}
\newtheorem{notation}[theorem]{Notation}
\newtheorem{observation}[theorem]{Observation}
\newtheorem{assumption}[theorem]{Assumption}
\newtheorem{hypothesis}[theorem]{Hypothesis}
\newtheorem{conjecture}[theorem]{Conjecture}
\newtheorem{example}[theorem]{Example}
\newtheorem{property}[theorem]{Property}
\newtheorem{condition}[theorem]{Condition}
\newtheorem{corollary}[theorem]{Corollary}
\theoremstyle{definition} %remarks in rm
\newtheorem{remark}[theorem]{Remark}

%\bibliographystyle{JPE}
%\usepackage{cite}

\usepackage{stmaryrd}
\def\scomm#1#2{\ensuremath{\bigr\llbracket #1:#2\bigr\rrbracket}}

\newcommand{\meq}[1]{\ensuremath{\stackrel{\scriptscriptstyle#1}{\scriptstyle
			\sim}}}


\newcommand{\dd}{\mathrm{d}}
\newcommand{\dt}{\,\dd t}
\newcommand{\mpart}[2]{\frac{\partial #1}{\partial #2}}
\newcommand{\avg}[1]{\left\langle #1\right\rangle}

\newcommand{\bigoh}[1]{\hat {\mathcal O}(p_2^{#1})}
\newcommand{\bigohneg}[1]{\hat {\mathcal O}\!\left({p_2^{-#1}}\right)}

\newcommand{\jp }[1]{{\color{magenta}jp*****:  #1}}
\newcommand{\noe }[1]{{\color{red}noe:  #1}}
\newcommand{\christophe }[1]{{\color{red}christophe: #1}}
\def\argcdot{{\,\cdot\,}}


\newcommand{\cA}{{\ensuremath{\mathcal A}} }
\newcommand{\cF}{{\ensuremath{\mathcal F}} }
\newcommand{\cP}{{\ensuremath{\mathcal P}} }
\newcommand{\cE}{{\ensuremath{\mathcal E}} }
\newcommand{\cH}{{\ensuremath{\mathcal H}} }
\newcommand{\cC}{{\ensuremath{\mathcal C}} }
\newcommand{\cN}{{\ensuremath{\mathcal N}} }
\newcommand{\cL}{{\ensuremath{\mathcal L}} }
\newcommand{\cT}{{\ensuremath{\mathcal T}} }
\newcommand{\cD}{{\ensuremath{\mathcal D}} }
\newcommand{\cU}{{\ensuremath{\mathcal U}} }
\newcommand{\cV}{{\ensuremath{\mathcal V}} }
\newcommand{\cS}{{\ensuremath{\mathcal S}} }
\newcommand{\cY}{{\ensuremath{\mathcal Y}} }


%%%%%%%%%%%%%%%%%%%%%%%%%%%%%%%%%%%%%%%%%%%%%%%%%%%%%%%%%%%%%%%%%%%%%%%%%%%%%%
%%%%%%%%%%%% Blackboard bolds
%%%%%%%%%%%%%%%%%%%%%%%%%%%%%%%%%%%%%%%%%%%%%%%%%%%%%%%%%%%%%%%%%%%%%%%%%%%%%%

\newcommand{\bbA}{{\ensuremath{\mathbb A}} }
\newcommand{\bbB}{{\ensuremath{\mathbb B}} }
\newcommand{\bbC}{{\ensuremath{\mathbb C}} }
\newcommand{\bbD}{{\ensuremath{\mathbb D}} }
\newcommand{\bbE}{{\ensuremath{\mathbb E}} }
\newcommand{\bbF}{{\ensuremath{\mathbb F}} }
\newcommand{\bbG}{{\ensuremath{\mathbb G}} }
\newcommand{\bbH}{{\ensuremath{\mathbb H}} }
\newcommand{\bbI}{{\ensuremath{\mathbb I}} }
\newcommand{\bbJ}{{\ensuremath{\mathbb J}} }
\newcommand{\bbK}{{\ensuremath{\mathbb K}} }
\newcommand{\bbL}{{\ensuremath{\mathbb L}} }
\newcommand{\bbM}{{\ensuremath{\mathbb M}} }
\newcommand{\bbN}{{\ensuremath{\mathbb N}} }
\newcommand{\bbO}{{\ensuremath{\mathbb O}} }
\newcommand{\bbP}{{\ensuremath{\mathbb P}} }
\newcommand{\bbQ}{{\ensuremath{\mathbb Q}} }
\newcommand{\bbR}{{\ensuremath{\mathbb R}} }
\newcommand{\bbS}{{\ensuremath{\mathbb S}} }
\newcommand{\bbT}{{\ensuremath{\mathbb T}} }
\newcommand{\bbU}{{\ensuremath{\mathbb U}} }
\newcommand{\bbV}{{\ensuremath{\mathbb V}} }
\newcommand{\bbW}{{\ensuremath{\mathbb W}} }
\newcommand{\bbX}{{\ensuremath{\mathbb X}} }
\newcommand{\bbY}{{\ensuremath{\mathbb Y}} }
\newcommand{\bbZ}{{\ensuremath{\mathbb Z}} }


%%%%%%%%%%%%%%%%%%%%%%%%%%%%%%%%%%%%%%%%%%%%%%%%%%%%%%%%%%%%%%%%%%%%%%%%%%%%%%
%%%%%%%%%%%% Greek letters
%%%%%%%%%%%%%%%%%%%%%%%%%%%%%%%%%%%%%%%%%%%%%%%%%%%%%%%%%%%%%%%%%%%%%%%%%%%%%%

\newcommand{\ga}{\alpha}
\newcommand{\gb}{\beta}
\newcommand{\gga}{\gamma}            % \gg already exists...
\newcommand{\gd}{\delta}
\newcommand{\gep}{\varepsilon}       % \ge already exists...
\newcommand{\gp}{\varphi}
\newcommand{\gr}{\rho}
\newcommand{\gvr}{\varrho}
\newcommand{\gz}{\zeta}
\newcommand{\gG}{\Gamma}
\newcommand{\gP}{\Phi}
\newcommand{\gD}{\Delta}
\newcommand{\gk}{\kappa}
\newcommand{\go}{\omega}
\newcommand{\gto}{{\tilde\omega}}
\newcommand{\gO}{\Omega}
\newcommand{\gl}{\lambda}
\newcommand{\gL}{\Lambda}
\newcommand{\gs}{\sigma}
\newcommand{\gS}{\Sigma}
\newcommand{\gt}{\vartheta}
\let\kappa=\varkappa
\let\phi=\varphi
\def\CC{{\mathcal C}}
\def\Rsch{r_{\rm sch}}
\def\d{{\rm d}} 
\def\KK{{\mathcal K}}
\def\OO{{\mathcal O}}
\def\integer{{\mathbb Z}}
\def\real{{\mathbb R}}
\newcommand{\ind}{\mathbf{1}}
\def\p2t2{{\tilde p_2^{\,2}}}

\def\fhc#1{\textcolor{violet}{#1}}
\definecolor{bittersweet}{rgb}{1.0, 0.44, 0.37}
\newcommand{\fhn}[1]{{\textcolor{bittersweet}{\sf[#1]}}}
\newcommand{\de}[1]{\text{DE}}

\newcommand{\edit}[2] {\textcolor{black}{\sout{#1}} {\textcolor{violet}{#2}} }
\def\JP#1{\textcolor{blue}{JPE:#1}}
\let\epsilon=\varepsilon
\usepackage{authblk}

%%%%%%%
%% Farbod Hassani
%%%%%%%
\def\be{\begin{equ}}
\def\ee{\end{equ}}
\def\ww{,w}
\def\ww{}

\def\citep#1{\cite{#1}}
%%%%%% the omegas
\def\Ode{\Omega_{\rm DE}}
\def\Om{\Omega_{\rm M}}
\def\Or{\Omega_{\rm R}}

\newcommand{\HH}{\mathcal {H}}
\newcommand{\NN}{\mathcal {N}}
\newcommand{\LL}{\mathcal {L}}
\def\atau{\bm a(\tau )}
\usepackage{bm}

\begin{document}

\title{...}
\author[1,2]{...}
\author[2]{...}


\affil[1]{D\'epartement de Physique Th\'eorique and Section de
  Math\'ematiques, University of Geneva, Switzerland}
%  , $^*$corresponding author
%, jean-pierre.eckmann@unige.ch}
\affil[2]{D\'epartement de Physique Th\'eorique, University of Geneva, Switzerland}
\maketitle
\JP{made from source file \jobname}

\JP{ missing citations are always $\backslash$cite\{xxx\}}

\abstract{ In this paper we study the blowup phenomena in cosmological models,
  especially in 1+1 dimensions. We obtain ordinary differential
  equations for the evolution of the dark energy scalar field shape
  (considered in the form $a x + bx^2 + c x^3+\dots$ ) derived from
  the partial differential equation for the $k$-essence dark energy
  model. Then we solve this system in a cosmological scheme and comment
  on the evolution of the system for different choices of
  parameters. This analysis is especially important for studying of
  similar non-linear systems appearing in the EFT language. In
  particular, extending the results of \cite{xxx,xxx}, we exhibit
  that, depending on the speed of sound 
  $c_s$, and the equation of state $w$, the equations can diverge in
  finite time. This limits the set of feasible parameters in the class
  of models we consider. 
  
} 
  


 \section{Introduction}
 Effective field theories (EFT) are employed to describe different fundamental theories at low energy limit in a unified form (Ref XXX). For the cosmological applications we use the EFT language for the dark energy scalar field in the weak field regime. In this framework the dark energy field and 
gravity part are coupled and are conveniently described by non-linear partial
differential equations. Extensive numerical simulations of EFT of dark energy are done using $k$-evolution $N$-body code which is developed based on gevolution (a general relativistic $N$-body code) \cite{xxx}. In $k$-evolution the whole framework and the PDEs are discussed, and in XXX have shown that the solutions of
such equations can form violent singularities at \emph{finite} time,
at a redshift which happens before the current epoch of the universe.
Obviously such a theory should be rejected in the EFT framework.

In the paper \citep{xxx}, it was shown that the main cause of such
finite-time divergence is to be found in the equation
\begin{equ}\label{eq:pidd}
  \ddot \pi =\beta\cdot (\pi')^2~,
\end{equ}
where $\beta$ is some physical constant. In particular, if
$\pi$ is of the form $\pi(t,x)=\beta\cdot  \alpha(t)x^2$ then this
leads to $\alpha ''(t)=4 \beta  \cdot \alpha '(t)^2$,
whose general solution is
\begin{equ}\label{eq:aaprime}
  \alpha '(t)=\frac{1}{t_0-4\beta t}~,
\end{equ}
when $\alpha '(0)=1/t_0$. Clearly, such solutions diverge at
$t=\frac{t_0}{4\beta }$
when $\beta >0$. While this naive example illustrates the main
mechanism of divergence (in $1+1$ dimensions) it seems that this is at
the source of the divergences in $1+3$ dimensions observed in
\citep{xxx}.

Of course \eref{eq:aaprime} is only a poor caricature of the equations
used in cosmological calculations, which have many more terms, such as
\begin{equa}\label{eq:full}
& \ddot{\pi}(x,\tau) +  \LL (\pi, \dot \pi, \pi'', \Psi)  =  \NN (\pi,  	\dot \pi,   \pi',    \dot \pi',  \pi'', \Psi, \Psi') \,, 
\end{equa}
with the linear part given by 
\begin{equa}[eq:linear]
 \LL (\pi, \dot \pi, \pi'', \Psi)  & =  \bm \HH(\tau) (1-3w) \partial_{\tau}\pi(x,\tau)\nonumber\\&  + \Big( \partial_{\tau} \bm \HH(\tau) - 3w \bm \HH(\tau)^2 + 3 c_{s}^{2} (\bm \HH (\tau)^{2} - \partial_{\tau} \bm \HH(\tau))  \Big)  \pi(x,\tau)   \nonumber \\ & 
-  \partial_{\tau}\Psi(x,\tau) +3\bm \HH(\tau) (w- c_s^2) \Psi(x,\tau) - 3 c_{s}^{2} \partial_{\tau}\Phi(x,\tau)-c_{s}^{2}  \partial_{x}^2\pi (x,\tau)~, 
\end{equa}
and the non-linear one by
\begin{equa}[eq:nonlinear]
&  \NN (\pi,  	\partial_\tau \pi,  \partial_x \pi,  \partial_x\partial_{\tau}  \pi,\partial_x^2 \pi) 
 =    - \frac{\bm \HH (\tau)}{2}  \big(5c_s^2  + 3w  -2\big) \; {(\partial_x \pi(x,\tau))^2}\\
 	&+ 2(1- c_s^2)  \partial_x \pi(x,\tau) \partial_x\partial_{\tau}  \pi(x,\tau)
	 	  \\ &
	 - \Big[ (c_s^2-1) \big( \partial_{\tau}  \pi(x,\tau) + {\bm\HH}(\tau) \pi(x,\tau) -\Psi(x,\tau) \big)
		 + c_s^2 (\Phi(x,\tau) - \Psi(x,\tau)) 
	 \\ &
	+ 3 {\bm \HH}(\tau) c_s^2 (1+w)\pi(x,\tau)  \Big] \partial_x^2 \pi (x,\tau)
	 + (2 c_s^2 -1)  \partial_x \Psi(x,\tau)  \partial_x \pi(x,\tau)   	  
	 \\ &
 - c_s^2 \partial_x \Phi(x,\tau)  \partial_x \pi(x,\tau) 
 +\frac{3(c_s^2 -1) }{2}  (\partial_x \pi(x,\tau))^2 \partial_x^2 \pi(x,\tau)~.
% 
\end{equa}
Here, $\bm \HH(\tau)$ is the physical time, expressed in terms of the
conformal time $\tau $, $c_s$ is the speed of sound and $w$ describes
the equation of state.
The fields $\Psi$ and $\Phi$ describe the Bardeen potential, but,
later on, we will set $\Phi=\Psi$, as explained in \cite{xxx}.
Such equations appear in the literature \cite{xxx} we now ask
whether the phenomenon of \eref{eq:aaprime} survives the addition of
the many non-linear terms to the term  $- \frac{\bm \HH (\tau)}{2}
\big(5c_s^2  + 3w  -2\big) \; {(\partial_x \pi(x,\tau))^2}$ which we
identified as the cause of divergence in \eref{eq:pidd} and \eref{eq:aaprime}.

While the technology for a general discussion of non-linear PDEs is
just not available, we will check in this paper that, indeed,
divergence \emph{will} appear at finite time, basically in the $w$,
$c_s^2$ region defined approximately by $w\ge (1-3c_s^2)/(1+c_s^2)$.
The reason for this stability is that the ``strength'' of these
additional terms (as we will show below) is weaker than the force
which leads to divergence. Unfortunately, this is not a mathematical
statement: The reason is that, while each term in itself can be
bounded, the theory of non-linear PDEs does not allow any conclusion
about the back-reaction this can have on the coupled equation.

In order to do this, we developed a Mathematica notebook for the model
described above. This allows, see below, to study the properties of the
model for initial densities $\rho$ of the xxx. But users can also
easily modify, if needed, the algorithm to other variants of the
model, such as can be found in \citep{xxx,xxx}.

Since the blowup of solutions is a local phenomenon, we expand the
(local) solution in terms of powers of the space coordinate,
$x$. Then, the system \eref{eq:full} decouples into a system of ODEs
for the coefficients of the various powers of $x$, for the fields
$\pi$ and $\Psi$.



\section{The effective field theory of dark energy in 1+1 D}
The full partial differential equation for the $k$-essence model as a
viable dark energy candidate is obtained in the effective field theory
framework in Eqs.~2.20 and 2.21 in \cite{Hassani:2019lmy}. According
to \cite{Hassani:2019lmy}, this PDE in 3+1 dimensions for some choice of
parameters is unstable and blows up in finite time. Writing the
equations in a second order equation form and considering it in 1+1
dimension results in the following second order partial differential
equation:
\begin{equa}
 \ddot{\pi}(x,\tau) +  \LL (\pi, \dot \pi, \pi'', \Psi)  =  \NN (\pi,  	\dot \pi,   \pi',    \dot \pi',  \pi'', \Psi, \Psi') \,, \label{full_equation}
\end{equa}

\begin{equa}[non_linear_terms]
&  \NN (\pi,  	\partial_\tau \pi,  \partial_x \pi,  \partial_x\partial_{\tau}  \pi,\partial_x^2 \pi) 
 =    - \frac{\bm \HH (\tau)}{2}  \big(5c_s^2  + 3w  -2\big) \; {(\partial_x \pi(x,\tau))^2}\\
 	&+ 2(1- c_s^2)  \partial_x \pi(x,\tau) \partial_x\partial_{\tau}  \pi(x,\tau)
	 	  \\ &
	 - \Big[ (c_s^2-1) \big( \partial_{\tau}  \pi(x,\tau) + {\bm \HH}(\tau) \pi(x,\tau) -\Psi(x,\tau) \big)
		 + c_s^2 (\Phi(x,\tau) - \Psi(x,\tau)) 
	 \\ &
	+ 3 \HH(\tau) c_s^2 (1+w)\pi(x,\tau)  \Big] \partial_x^2 \pi (x,\tau)
	 + (2 c_s^2 -1)  \partial_x \Psi(x,\tau)  \partial_x \pi(x,\tau)   	  
	 \\ &
 - c_s^2 \partial_x \Phi(x,\tau)  \partial_x \pi(x,\tau) 
 +\frac{3(c_s^2 -1) }{2}  (\partial_x \pi(x,\tau))^2 \partial_x^2 \pi(x,\tau)
 \end{equa}
and the linear terms $ \LL (\pi, \dot \pi, \pi'', \Psi)$ read as,
\begin{equa}[Linear_terms]
 \LL (\pi, \dot \pi, \pi'', \Psi)  & = + \bm \HH(\tau) (1-3w) \partial_{\tau}\pi(x,\tau)\\&  + \Big(  \bm \HH'(\tau) - 3w \bm \HH(\tau)^2 + 3 c_{s}^{2} (\bm \HH (\tau)^{2} - \bm \HH'(\tau))  \Big)  \pi(x,\tau)    \\ & 
-  \partial_{\tau}\Psi(x,\tau) +3\bm \HH(\tau) (w- c_s^2) \Psi(x,\tau)\\& - 3 c_{s}^{2} \partial_{\tau}\Phi(x,\tau)-c_{s}^{2}  \partial_{x}^2\pi (x,\tau) \label{Linear_terms}
\end{equa}
\subsection{The background equations}


In this subsection we discuss some of the terms appearing in  \eref{full_equation}, namely $\bm \HH(\tau), \bm \HH'(\tau)$ and $\Psi(\tau,x), \Psi'(\tau,x)$. 
To keep the discussion clear, we express, whenever possible, all
time-dependent quantities as functions of the conformal time $\tau
$. The well-known relations with the redshift $z$ are explained in
\aref{sec:timescales}, but keep in mind that this relation depends on
$w$, and we will take this into account. We also express $\bm \HH$ as a function of the
logarithmic derivative $\bm a$ (the scale factor).
Looking at \aref{sec:timescales} \eref{eq:t}, we immediately deduce
that $\bm H'(\tau)$ can be computed from the knowledge of $\bm a$.



To find $\bm \HH(\tau), \bm \HH'(\tau)$ we use the Friedmann equations, 
\be
\bm \HH (\tau)^2 = \HH_0^2 \Big( \atau ^{-(1 + 3 w)}\Ode +  
\, \atau ^{-1}\Om+  \atau ^{-2}\Or    \Big)~. \label{Friedmann_eq}
\ee

The cosmological values of the $\Omega$'s are given by
$\Ode = 0.687$, $\Om  = 0.31$, and $\Or = 9 \times
10^{-5}$. Although $\Or$ is quite small, we keep it in our programs,
but the reader can easily set it to 0 in the Mathematica notebook.
Moreover $\HH_0$ equals
$67 \; \text{[Km/s/Mpc]}$ which in our code units, i.e., $c=1$ and
measuring time in Giga light years leads to
\begin{equa}\HH_0 &=\frac{ 67  }{299792.458 \; \text{ [Km/s]} \times
    0.003261 \text{[Mpc to Giga year conversion]} }\\&= 0.0691023 \;
  \text{ [Giga light-year]$^{-1}$}~.
\end{equa}


It is important to note that our parameter is the conformal time
$\tau$, however we want to set the initial and final condition based
on another parameter called redshift and it is defined as  $\bm
z(\tau) \equiv 1/\atau -1$ and the conformal time for a given redshift is
\begin{equ}\label{tau_equ}
\bm\tau(z) =\HH_{0}^{-1} \int_{z}^{\infty} \frac{\d y}{\sqrt{  (1+y)^{3 (1+w)}\Ode+ (1+y)^{3}\Om+ (1+y)^{4}\Or}}~.
\end{equ}

So for example, for $\HH_0^{-1} = 14.4713$, when $w=-1$ and if we
choose $z = 0$ corresponding to the present time,
\begin{equa}
\tau_{\,\rm f} &= 14.4713 \int_{z = 0}^{\infty} \frac{\d y}{\sqrt{  (1+y)^{3 (1+w)}\Ode+ (1+y)^{3}\Om+\ (1+y)^{4}\Or}}\\&= 46.3334 \; \text{[Giga-light year]}~.
\end{equa}

%Note that the  definition of $\bm\tau(z)$ in \eref{tau_equ} is consistent with the Friedmann equation \eref{Friedmann_eq} and the definition of the redshift as,
%\be
%\frac{\partial \bm \tau(z)}{\partial z} =  -\HH_0^{-1} \frac{1}{\sqrt{\Ode  (1+z)^{3 (1+w)}+\Om (1+z)^{3}+\Or (1+z)^{4}}}~.
%\ee
%%\JP{according to eckmann-rules, we cannot use the same symbol for
%  $\HH$ as a function of $\tau$ or $z$. But I think we do not really
%  need it because of the appendix.}

While our calculation are all done in terms of the conformal time
$\tau $, it is common to determine epochs in terms of the redshift $z$.
The relation between the two variables is given by the Friedmann
equations as follows (see \aref{sec:timescales} for details):
The functions $z\mapsto \bm\tau(z)$ and $\tau\mapsto \bm z(\tau)$ are
inverses of each other and are related through the \emph{scale factor}
$\bm a$ defined by the Friedman equations:
\begin{equ}
  \bm a'(\tau )=\HH_0 \sqrt{ \atau
     ^{1-3w}\,\Ode+ \atau\,\Om +\Or}~,
\end{equ}
with
\begin{equ}
  \bm z (x)=\frac{1}{\bm a(x)}-1~.
\end{equ}

The
advantage of using redshift is that  we can control the initial
redshift  $z_{\rm i}$ and final redshift $z_{\,\rm f}$. Because we
want to start our simulation in matter dominated epoch $z_{\rm i}\sim1000$
while it continues to today $z_{\,\rm f} = 0$. For
different dark energy models (different $w$) $\tau_{\rm i}$ and
  $\tau_{\,\rm f}$ vary with $w$, but we want to keep the more natural 
$z_{\rm i}$ and $z_{\,\rm f}$ fixed.

\subsection{The equation for the gravitational potential}
Following the discussion of \cite{Bernardeau:2001qr}(Eq.~(22)) the linear growth factor for matter perturbation employing the Eulerian Linear Perturbation Theory reads,
\be
\frac{\mathrm{d}^{2} D_{1}(\tau)}{\mathrm{d} \tau^{2}}+\bm \HH (\tau) \frac{\mathrm{d} D_{1}(\tau)}{\mathrm{d} \tau}=\frac{3}{2} \Om\atau^{-3} \bm \HH ^{2}(\tau) D_{1}(\tau)~. \label{eq_growthfactor}
\ee
The growth factor for the gravitational potential $\Psi$ according to
the Poisson equation in Fourier space is $ -k^2 \Psi \sim \delta_m/a $, so we need to solve the equation for $D_1(\tau)$ to find the time evolution of $\bm \Psi(\tau) \sim D_1(\tau)/\atau $.
Equivalently we can also substitute $D_1(\tau) =\bm \Psi(\tau)  \atau $ in
\eref{eq_growthfactor} and obtain a (linear in $\bm \Psi$)  ODE for the time evolution of
$\bm \Psi$, which is
\begin{equa}
\bm \Psi''(\tau )+3\bm \HH(\tau )\bm \Psi'(\tau
)+\big((2-3\Om\atau^{-3}/2)\bm \HH(\tau )^2 +\bm \HH'(\tau) \big)\bm \Psi(\tau )=0~.
\end{equa}

\JP{from your notebook, rearranged, please check}

To solve this  ODE, we assume that at the initial time
$\bm \Psi(\tau_{\,\rm i}) = k$ and $\bm \Psi'(\tau_{\,\rm i})=0$  where $k$ is a constant.
This also implies that at the initial time $D_1(\tau) = k \atau$. As
a result we can solve the second order ODE and find the solution of
$\bm \Psi$ at every time $\tau$.
\fhn{This equation should be checked! with the previous results.}
%  \JP{next still incomplete}
%
%To find $\Psi$ and $\Psi'$ we define a new variable $N = \ln a$ to simplify the equations. The Friedmann equation \eref{Friedmann_eq} reads,
%\be 
% \bm \HH ^2 = \bm \HH_0^2  \Big(  \Ode  e^{-(1 + 3 w)N }+ \Om  e^{-N} + \Or  e^{-2N}  \Big)
%\ee 
%Here for simplicity we neglect radiation contribution which is consistent with the fact that we take the initial redshift in the matter dominated era,  therefore
%\be
%\partial_N \ln\bm \HH  = \frac{\partial_N \left(\bm \HH ^2\right)}{2 \bm \HH ^2} = -\frac{\Om  + \left(1 + 3 w\right) \Ode  e^{-3 w N}}{2 \Om  + 2 \Ode  e^{-3 w N}}\,.
%\ee
%
%If we neglect gravitational back-reaction from dark energy and assume that $\Psi$ is generated by a linear matter perturbation, we can also write the time evolution equation for $\Psi$, \fhn{Reference and also do not  remove $\Omega_r$ from the previous equation.}
%
%\be 
% \partial_N^2 \Psi + \left(3 + \partial_N \ln \bm \HH \right) \partial_N \Psi + \left(2 - \frac{3\Om }{2\Om  + 2 \Ode  e^{-3 w N}} + \partial_N \ln\bm \HH \right) \Psi = 0\,.
%\ee 
%We assume $\Psi (N_i) =  k$ and $\frac{\partial{\Psi}}{DNA}(N_i)= 0$.



\section{The simplified equation}

In  [Peter's papers]  we studied;
\be
\ddot{\pi}(x,\tau) = \pi'(x,\tau)^2\,,
\ee
where dot denotes the derivative with respect to time $\tau$ and prime denotes the spatial derivative. Here, in addition, we consider the other relevant linear and non-linear terms in our analysis 

For  simplicity in our analysis we define the coefficients of \eref{Linear_terms} as:
\begin{equa}
 \LL (\pi, \dot \pi, \pi'', \Psi)  & = \ell_{\dot{\pi}} (\tau)  \dot{\pi}(x,\tau)  +  \ell_{{\pi}} (\tau) \pi(x,\tau)\\&+   \ell_{{\Psi}} (\tau) \Psi(x,\tau)  +  \ell_{{\dot\Psi}} (\tau) \dot\Psi(x,\tau)  +  \ell_{{\pi''}}(\tau)  \pi'' (x,\tau) ~,
\end{equa}
where we have (2 coefficients are independent of $\tau $): 
\begin{equa}[eq:ell]
\ell_{\dot{\pi}} (\tau)  &=  (1-3w) \bm \HH(\tau)~, \\ 
 \ell_{{\pi}} (\tau)  &= (1-3c_s^2)  \bm \HH'(\tau) + 3
 (c_s^2-w)  \bm \HH(\tau)^2 ~,\\
 \ell_{{\Psi}} (\tau)  &= 3 (w- c_s^2) \bm\HH (\tau)~, \\
  \ell_{{\dot\Psi}}   &= - (1+3 c_s^2)~,   \\
 \ell_{{\pi''}}  &= -c_{s}^{2}~. \\
\end{equa}

In a general EFT framework the non-linear function in \eref{non_linear_terms}  can be a complicated function of many non-linear terms. However in the EFT of DE framework  for the $k$-essence theory only some non-linear terms appear in the equations:
\begin{equa}
  \NN (\pi,  	\dot \pi,   \pi',    \dot \pi',  \pi'') &
  =  \nu_{\pi'^2} (\tau) \,  \pi'^2
  +  \nu_{\pi' \dot \pi'} (\tau)  \, \pi' \dot \pi'
  +  \nu_{\pi \pi''} (\tau) \, \pi \pi''
  +   \nu_{\dot \pi \pi''} (\tau) \,\dot \pi \pi''   \\ & 
 +  \nu_{ \Psi \pi''} (\tau) \, \Psi \pi'' +  \nu_{\Psi' \pi' } (\tau)\, \Psi' \pi' +  \nu_{\pi'^2 \pi''} (\tau) \, \pi'^2 \pi''.
\end{equa}
where each coefficient reads (some do not depend on $\tau$);
\begin{equa}[eq:nu]
\nu_{\pi'^2} (\tau) &= -\frac{\bm \HH(\tau)}{2} (5 c_s^2 + 3w -2)= -\frac{\bm \HH(\tau)}{2} \big(5( c_s^2-1) + 3(1+w)\big)~,\\
 \nu_{\pi' \dot \pi'} &= 2 (1- c_s^2) ~,\\
 \nu_{\pi \pi''} (\tau) &= -\bm \HH(\tau)\Big[(c_s^2-1) + 3c_s^2 (1+w)\Big] ~,\\
\nu_{\dot \pi \pi''}   &= - (c_s^2 -1)~,\\
\nu_{ \Psi \pi''}  &= (c_s^2 -1)~,\\
\nu_{\Psi' \pi' }  &= (c_s^2 -1)~,\\
\nu_{\pi'^2 \pi''}  &= \frac{3}2 (c_s^2 -1)~.
\end{equa}
Note that the terms $(1-c_s^2)$ and $(1+w)$ are the two important
parameters that control the non-linear part in the evolution of the scalar field. The speed
of sound squared $c_s^2$ and the equation of state $w$ can take the values;
\begin{equa}
 -1.5 \leq w \leq -0.5 ~,\qquad 
 0 \leq c_s^2 \leq 1~.
\end{equa}


The initial conditions for this PDE are set to 0 at the initial time as we want it to be generated by the gravitational potential at early times (here early means compared to the dark energy domination era). However we also vary the initial conditions for the scalar field a bit  in our analysis to see the impact of non-zero initial condition \fhn{To be done!}. 

The
gravitational potential is the main quantity which acts as the source for the scalar field $\pi$, we assume that $\Psi(x,\tau)$ is a polynomial at the initial time ($z\sim 1000$):
\be
\Psi(x) = \psi_0(\tau) + \psi_1(\tau)  x +  \psi_2(\tau)  x^2 +  \psi_3(\tau)  x^3 +  \psi_4(\tau)  x^4~.
\ee
In general, $\Psi$ is a stochastic field and has  peaks and troughs in space. A good approximation around the minima and maxima of $\Psi$ would be to consider 
 $\Psi(x) = \psi_2(\tau)  x^2$ where the sign of $ \psi_2(\tau)$ shows
whether it is a structure or void, a positive $ \psi_2(\tau)$ corresponding to a
structure and a negative one to a void. Moreover in our analysis we
took the speed of light $c =1$ and we measure the conformal time in
giga years which implies that the distances are measured in giga
light-years (the distance that light travels in one interval of
time (here giga year)). According to our choice of units  conformal
time $\tau$ varies from a small number, for example $\tau_{\rm i} = 0.01$
to $\tau_{\,\rm f} = 47 $.
Note that the exact values of $\tau_{\rm i}$ and $\tau_{\,\rm f}$ are
found by fixing $z_{\rm i} =1000$ and $z_{\rm f}=0$ and finding the
corresponding $\tau$ (which depends on $w$, see \eref{eq:tau}).
To solve \eref{full_equation} we consider the following ansatz;
\be 
\pi(\tau,x) = u_0(\tau)+  u_1(\tau) x+   u_2(\tau) x^2 + u_3(\tau) x^3 + u_4(\tau) x^4 + ...\,.
\ee
We use this ansatz and keep all the terms up to $x^4$ in the equation to
obtain ODEs out of the PDE for the evolution of each coefficient. The
equations one obtains in this way are displayed in \aref{sec:full}.


\section{Conclusions}


\section*{Acknowledgments}
%We thank Martin Kunz for asking questions leading to this paper, and
%for helpful discussions. We thank Ruth Durrer and Jacques Rougemont
%for helpful comments about our manuscript. JPE acknowledges partial support
%by an ERC advanced grant ``Bridges'' and FH acknowledges financial support from the Swiss National Science Foundation.
%


\newpage
\appendix 

\section{Time scales}\label{sec:timescales}

Let us assume $w$ is fixed. Therefore, we can write, $a(\tau
,w)=a(\tau)$ but it is understood that all functions depend on $w$.
Then there are 3 variables which relate to
times: One is the \emph{redshift}, $z$. The \emph{conformal time}
$\bm{\tau}$, as a function of redshift is given by
\begin{equ}\label{eq:tau}
  \bm{\tau}(z\ww )=\frac{1}{\HH_0}\int_z^\infty \kern -0.3 em\d y
  \frac{1}{\sqrt{ (1+y)^{3(1+w)}\,\Ode+ (1+y)^3\,\Om+(1+y)^4\,\Or}}~.
\end{equ}
We use this in the differentiated form
\begin{equ}\label{eq:tau2}
  \bm{\tau}'(z\ww )=-\frac{1}{\HH_0}
  \frac{1}{\sqrt{(1+z)^{3(1+w)}\,\Ode+(1+z)^3\,\Om+(1+z)^4\,\Or}}~.
\end{equ}
Note that here, and below, the $'$ always denotes the derivative of
the function (even if the function is evaluated at a composed argument).
By definition $\bm\tau (0)=z_{\rm f}$, the (conformal) current time, while we
consider $\bm\tau(1000)=z_{\rm i}$, the time when the theory begins.

Let now $\bm z$ be the inverse function of $\bm \tau $, this the
\emph{redshift}:
\begin{equ}
  \bm \tau(\bm z (x))=x~.
\end{equ}
Then
\begin{equ}\label{eq:zinverse}
\bm \tau'(\bm z (x))\cdot \bm z' (x)=1~,
\end{equ}
 for all $x$.  
We now define $\bm a$ by
\begin{equ}\label{eq:adef}
  \bm z (x)=\frac{1}{\bm a(x)}-1~.
\end{equ}
Therefore, combining \eref{eq:adef} and \eref{eq:zinverse}, we get
\begin{equa}
\bm z'(x)&=-\frac{\bm a'(x)}{\bm a(x)^2}~,\\
\bm z'(x)&=\frac{1}{\bm\tau' (\bm z(x))}= \frac{1}{\bm \tau '(1/\bm a(x)-1)}~.
\end{equa}
which is the same as
\begin{equ}
  \bm a'(x)=-\frac{\bm a(x)^2}{\bm \tau '(1/\bm a(x)-1)}~.
\end{equ}
Substituting into \eref{eq:tau2} we obtain
the Friedmann equations:
\begin{equ}\label{eq:aprimea}
  \bm a'(\tau )=\HH_0 \sqrt{ \atau
     ^{1-3w}\,\Ode+ \atau\,\Om +\Or}~,
\end{equ}
with initial condition $\bm a(\bm{\tau}(0) )=1$.
The \emph{physical time}, $\bm \HH$, is defined by
\begin{equ}\label{eq:t}
  {\bm \HH}(\tau  )=\frac{\bm a'(\tau  )}{\atau}~.
\end{equ}
As we will also need a handle on $\bm \HH'$, we write it as
\begin{equ}\label{eq:hhprime}
  \bm \HH'(\tau)= \frac{\bm a''(\tau)}{\atau }-{\bm
    \HH}(\tau)^2~.
\end{equ}
From the equation for $\bm a'$ we get
\JP{check \eref{eq:hprime}}
\begin{equa}[eq:hprime]
 \bm \HH'(\tau)=&
  -\frac{\HH_0^2}{4}\Bigl((4\atau +3w-1)\atau^{-(2+3w)}\Ode\\
\\&+ ( 4\atau^{-1}-\atau^{-2}) \Om+4\atau^{-3}\Or\Bigr)~.
\end{equa}


%\JP{we have to decide at some point whether what follows is useful.}
%\JP{for the moment I dont see why we need this}
%
%It can also be expressed in terms of $z$, in which case we write
%\begin{equ}
%  \hat{\bm \HH}(z )=\frac{\bm a'(\bm\tau (z ) )}{\bm a(\bm\tau (z ) )}~,
%\end{equ}
%the $\hat{\ }$ tells us that we view $\bm\HH$ as a function of $z$, not $\tau$.
%From \eref{eq:t} we have
%\begin{equ}
%  \int_u^v \d \tau \,{\bm \HH}(\tau  )=\frac{1}{\bm a(v
%    )}-\frac{1}{\bm a(u )}
%\end{equ}
%and since $\bm a(\bm{\tau}(0) )=1$ we also have
%\begin{equ}\label{eq:aa}
%  \int_{\bm{\tau}(0)}^{\bm{\tau}(z )} \d y \, {\bm \HH}(y
%  )=\frac{1}{\bm a(\bm{\tau}(z ) )}-1~.
%\end{equ}
%

\section{The full equations}\label{sec:full}
\def\mytau{}

Here, we write the equations for the coefficients of the powers of
$x$. The same equations are found in the Mathematica notebook.
All functions $u_0, u_1, \dots$ as well as all coefficients $\ell_{x}$
and $\nu_x$ depend in principle on $\tau$. However, for the model we
discuss here in detail, some of the coefficients only depend on $c_s^2$
and on $w$, see \eref{eq:ell} and \eref{eq:nu}.




\JP{if we really keep this, we have to re-align}



\begin{equa}
u_0''=-(&\ell_{\Psi }\mytau+\ell_{{\dot \Psi}}\mytau {\psi_{0}}'\mytau+\ell_{{\dot \pi}}\mytau u_0'\mytau+2 u_2\mytau \ell_{{\pi''}}\mytau+\ell_{\pi }\mytau u_0\mytau-2 u_2\mytau \,\nu_{{\dot \pi\pi''}}\mytau u_0'\mytau\\&-u_1\mytau^2 \,\nu_{\pi'^2}\mytau-2 u_0\mytau u_2\mytau \,\nu_{{\pi \pi''}}\mytau-u_1\mytau \,\nu_{{\pi'\dot\pi'}}\mytau u_1'\mytau-2 u_2\mytau u_1\mytau^2 \,\nu_{{\pi'^2\pi''}}\mytau-2 {\psi_{0}}\mytau u_2\mytau \,\nu_{{\Psi \pi''}}\mytau-{\psi_{1}}\mytau u_1\mytau \,\nu_{{\Psi' \pi'}}\mytau)~,\\
u_1''=-(&\ell_{\Psi }\mytau+\ell_{{\dot \Psi}}\mytau {\psi_{1}}'\mytau+\ell_{{\dot \pi}}\mytau u_1'\mytau+6 u_3\mytau \ell_{{\pi''}}\mytau+\ell_{\pi }\mytau u_1\mytau-6 u_3\mytau \,\nu_{{\dot \pi\pi''}}\mytau u_0'\mytau-2 u_2\mytau \,\nu_{{\dot \pi\pi''}}\mytau u_1'\mytau\\&-4 u_2\mytau u_1\mytau \,\nu_{\pi'^2}\mytau-2 u_2\mytau u_1\mytau \,\nu_{{\pi \pi''}}\mytau-6 u_0\mytau u_3\mytau \,\nu_{{\pi \pi''}}\mytau-2 u_1\mytau \,\nu_{{\pi'\dot\pi'}}\mytau u_2'\mytau-2 u_2\mytau \,\nu_{{\pi'\dot\pi'}}\mytau u_1'\mytau\\&-6 u_3\mytau u_1\mytau^2 \,\nu_{{\pi'^2\pi''}}\mytau-8 u_2\mytau^2 u_1\mytau \,\nu_{{\pi'^2\pi''}}\mytau-6 {\psi_{0}}\mytau u_3\mytau \,\nu_{{\Psi \pi''}}\mytau\\&-2 {\psi_{1}}\mytau u_2\mytau \,\nu_{{\Psi' \pi'}}\mytau-2 {\psi_{1}}\mytau u_2\mytau \,\nu_{{\Psi \pi''}}\mytau-2 {\psi_{2}}\mytau u_1\mytau \,\nu_{{\Psi' \pi'}}\mytau)~,\\
u_2''=-(&\ell_{\Psi }\mytau+\ell_{{\dot \Psi}}\mytau {\psi_{2}}'\mytau+\ell_{{\dot \pi}}\mytau u_2'\mytau+12 u_4\mytau \ell_{{\pi''}}\mytau+\ell_{\pi }\mytau u_2\mytau-2 u_2\mytau \,\nu_{{\dot \pi\pi''}}\mytau u_2'\mytau-12 u_4\mytau \,\nu_{{\dot \pi\pi''}}\mytau u_0'\mytau-6 u_3\mytau \,\nu_{{\dot \pi\pi''}}\mytau u_1'\mytau\\&-4 u_2\mytau^2 \,\nu_{\pi'^2}\mytau-6 u_1\mytau u_3\mytau \,\nu_{\pi'^2}\mytau-2 u_2\mytau^2 \,\nu_{{\pi \pi''}}\mytau-6 u_1\mytau u_3\mytau \,\nu_{{\pi \pi''}}\mytau-12 u_0\mytau u_4\mytau \,\nu_{{\pi \pi''}}\mytau-4 u_2\mytau \,\nu_{{\pi'\dot\pi'}}\mytau u_2'\mytau\\&-3 u_3\mytau \,\nu_{{\pi'\dot\pi'}}\mytau u_1'\mytau-3 u_1\mytau \,\nu_{{\pi'\dot\pi'}}\mytau u_3'\mytau-8 u_2\mytau^3 \,\nu_{{\pi'^2\pi''}}\mytau-36 u_1\mytau u_3\mytau u_2\mytau \,\nu_{{\pi'^2\pi''}}\mytau-12 u_1\mytau^2 u_4\mytau \,\nu_{{\pi'^2\pi''}}\mytau\\&-12 {\psi_{0}}\mytau u_4\mytau \,\nu_{{\Psi \pi''}}\mytau-3 {\psi_{1}}\mytau u_3\mytau \,\nu_{{\Psi' \pi'}}\mytau-6 {\psi_{1}}\mytau u_3\mytau \,\nu_{{\Psi \pi''}}\mytau-4 {\psi_{2}}\mytau u_2\mytau \,\nu_{{\Psi' \pi'}}\mytau\\&-2 {\psi_{2}}\mytau u_2\mytau \,\nu_{{\Psi \pi''}}\mytau-3 {\psi_{3}}\mytau u_1\mytau \,\nu_{{\Psi' \pi'}}\mytau)~,\\
u_3''=-(&\ell_{\Psi }\mytau+\ell_{{\dot \Psi}}\mytau {\psi_{3}}'\mytau+\ell_{{\dot \pi}}\mytau u_3'\mytau+\ell_{\pi }\mytau u_3\mytau-2 u_2\mytau \,\nu_{{\dot \pi\pi''}}\mytau u_3'\mytau-12 u_4\mytau \,\nu_{{\dot \pi\pi''}}\mytau u_1'\mytau\\&-6 u_3\mytau \,\nu_{{\dot \pi\pi''}}\mytau u_2'\mytau-12 u_3\mytau u_2\mytau \,\nu_{\pi'^2}\mytau-8 u_1\mytau u_4\mytau \,\nu_{\pi'^2}\mytau-8 u_3\mytau u_2\mytau \,\nu_{{\pi \pi''}}\mytau-12 u_1\mytau u_4\mytau \,\nu_{{\pi \pi''}}\mytau-6 u_2\mytau \,\nu_{{\pi'\dot\pi'}}\mytau u_3'\mytau\\&-4 u_4\mytau \,\nu_{{\pi'\dot\pi'}}\mytau u_1'\mytau-6 u_3\mytau \,\nu_{{\pi'\dot\pi'}}\mytau u_2'\mytau-4 u_1\mytau \,\nu_{{\pi'\dot\pi'}}\mytau u_4'\mytau-48 u_3\mytau u_2\mytau^2 \,\nu_{{\pi'^2\pi''}}\mytau\\&-64 u_1\mytau u_4\mytau u_2\mytau \,\nu_{{\pi'^2\pi''}}\mytau-36 u_1\mytau u_3\mytau^2 \,\nu_{{\pi'^2\pi''}}\mytau-4 {\psi_{1}}\mytau u_4\mytau \,\nu_{{\Psi' \pi'}}\mytau\\&-12 {\psi_{1}}\mytau u_4\mytau \,\nu_{{\Psi \pi''}}\mytau-6 {\psi_{2}}\mytau u_3\mytau \,\nu_{{\Psi' \pi'}}\mytau-6 {\psi_{2}}\mytau u_3\mytau \,\nu_{{\Psi \pi''}}\mytau-6 {\psi_{3}}\mytau u_2\mytau \,\nu_{{\Psi' \pi'}}\mytau\\&-2 {\psi_{3}}\mytau u_2\mytau \,\nu_{{\Psi \pi''}}\mytau-4 {\psi_{4}}\mytau u_1\mytau \,\nu_{{\Psi' \pi'}}\mytau+u_3)~,\\
u_4''=-(&\ell_{\Psi }\mytau+\ell_{{\dot \Psi}}\mytau {\psi_{4}}'\mytau+\ell_{{\dot \pi}}\mytau u_4'\mytau+\ell_{\pi }\mytau u_4\mytau-2 u_2\mytau \,\nu_{{\dot \pi\pi''}}\mytau u_4'\mytau-12 u_4\mytau \,\nu_{{\dot \pi\pi''}}\mytau u_2'\mytau-6 u_3\mytau \,\nu_{{\dot \pi\pi''}}\mytau u_3'\mytau\\&-16 u_4\mytau u_2\mytau \,\nu_{\pi'^2}\mytau-9 u_3\mytau^2 \,\nu_{\pi'^2}\mytau-14 u_4\mytau u_2\mytau \,\nu_{{\pi \pi''}}\mytau-6 u_3\mytau^2 \,\nu_{{\pi \pi''}}\mytau-8 u_2\mytau \,\nu_{{\pi'\dot\pi'}}\mytau u_4'\mytau-8 u_4\mytau \,\nu_{{\pi'\dot\pi'}}\mytau u_2'\mytau\\&-9 u_3\mytau \,\nu_{{\pi'\dot\pi'}}\mytau u_3'\mytau-80 u_4\mytau u_2\mytau^2 \,\nu_{{\pi'^2\pi''}}\mytau-90 u_3\mytau^2 u_2\mytau \,\nu_{{\pi'^2\pi''}}\mytau\\&-120 u_1\mytau u_3\mytau u_4\mytau \,\nu_{{\pi'^2\pi''}}\mytau-8 {\psi_{2}}\mytau u_4\mytau \,\nu_{{\Psi' \pi'}}\mytau-12 {\psi_{2}}\mytau u_4\mytau \,\nu_{{\Psi \pi''}}\mytau-9 {\psi_{3}}\mytau u_3\mytau \,\nu_{{\Psi' \pi'}}\mytau\\&-6 {\psi_{3}}\mytau u_3\mytau \,\nu_{{\Psi \pi''}}\mytau-8 {\psi_{4}}\mytau u_2\mytau \,\nu_{{\Psi' \pi'}}\mytau-2 {\psi_{4}}\mytau u_2\mytau \,\nu_{{\Psi \pi''}}\mytau)~.
\end{equa}


\JP{here, and in my programs, we should add the equations for $\Psi$}

\begin{thebibliography}{11}
\providecommand{\natexlab}[1]{#1}
\providecommand{\url}[1]{\texttt{#1}}
\providecommand{\urlprefix}{URL }

%\cite{Hassani:2019lmy}
\bibitem{Hassani:2019lmy}
F.~Hassani, J.~Adamek, M.~Kunz and F.~Vernizzi,
%``$k$-evolution: a relativistic N-body code for clustering dark energy,''
JCAP \textbf{12} (2019), 011
doi:10.1088/1475-7516/2019/12/011
[arXiv:1910.01104 [astro-ph.CO]].
%8 citations counted in INSPIRE as of 06 Jan 2021

%\cite{Bernardeau:2001qr}
\bibitem{Bernardeau:2001qr}
F.~Bernardeau, S.~Colombi, E.~Gaztanaga and R.~Scoccimarro,
%``Large scale structure of the universe and cosmological perturbation theory,''
Phys. Rept. \textbf{367} (2002), 1-248
doi:10.1016/S0370-1573(02)00135-7
[arXiv:astro-ph/0112551 [astro-ph]].
%1179 citations counted in INSPIRE as of 17 Jan 2021

\end{thebibliography}

\end{document}
